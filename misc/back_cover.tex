\newpage



\begin{titlepage}
    \begin{center}
    \LARGE \textbf{Dos du rapport}
    \vspace{0.7cm}
    \end{center}
    \par
    \raisebox{-.5\height}{Étudiant : Victor Malod}
    \hfill
    \raisebox{-.5\height}{Année d'étude dans la spécialité : INFO 4}
    
    \vspace{1cm}
    \par {Entreprise : Laboratoire Informatique de Grenoble}
    \par {Adresse : Bâtiment IMAG, 700 Av. Centrale, 38401 Saint-Martin-d'Hères}
    \par {Téléphone : 04 57 42 14 00}

    \vspace{1cm}
    \par {Responsable administratif : Pascale Poulet}
    \par {Téléphone : 04 57 42 14 01}
    \par {Courriel : pascale.poulet@imag.fr}

    \vspace{1cm}
    \par {Tuteur de stage : Nicolas Palix}
    \par {Téléphone : 04 57 42 15 38}
    \par {Courriel : nicolas.palix@imag.fr}

    \vspace{1cm}
    \par {Enseignant-référent : Vincent Danjean }
    \par {Téléphone : 04 57 42 14 76}
    \par {Courriel : vincent.danjean@imag.fr}

    \vspace{1cm}
    \par {Titre : \large Développement sur le compilateur de PhaistOS }

    \vspace{1cm}
\par {\underline{Résumé :}

Ce rapport de stage résume mon expérience de travail au sein du Laboratoire 
d'Informatique de Grenoble durant ma quatrième année d'études universitaires. 
Afin de valider mon année d'étude à l'école d'ingénieur de Polytech Grenoble, 
ce rapport sera fourni ainsi qu'une soutenance à un jury responsable de 
m'évaluer. 

Je faisais parti de l'équipe Erods et ma mission consistait à tester et 
améliorer le compilateur d'un langage de programmation dédié à l'ordonnancement 
d'entrées/ sorties d'un système linux. L'outil en question s'appelle PhaistOS 
et a été développé par Nick Papoulias, un chercheur postdoctoral qui ne travail 
maintenant plus sur le projet.

Durant ce stage différentes tâches m'ont été attribuées par mon tuteur, j'ai 
commencé par m'approprié le sujet en lisant la documentation et le code 
qu'avait écrit Nick. J'ai ensuite fait de la documentation puis du 
développement sur le compilateur. J'ai finis mon stage par des tests de 
performances et l'écriture d'un papier scientifique portant sur le compilateur.}

\end{titlepage}