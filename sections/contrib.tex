\section{La modernisation et les performances du Disque de Phaistos}
\label{contrib}

\begin{center}
    $\prec \cdot$ --- $\cdot \succ$ 
\end{center}

PhaistOS tire son nom de la découverte archéologique du disque de Phaistos, 
rappelant par son aspect un disque dur. Les lettres ``OS'' sont en majuscules 
pour rappeler le système d'exploitation (Operating System), une belle 
coincidence bien exploitée par Nick. 

\blockquote[Wikipédia][--]{\textit{Le disque de Phaistos ou disque de Phaestos 
est un disque d'argile cuite découvert en 1908 par l'archéologue italien Luigi 
Pernier sur le site archéologique du palais minoen de Phaistos, en Crète.}
}

\begin{center}
    $\prec \cdot$ --- $\cdot \succ$ 
\end{center}

Durant le long du stage sur ce projet, j'ai pu entretenir un journal 
récapitulant mon expérience quotidienne, ce journal est disponible au lien 
suivant :

\begin{center}
\href{https://github.com/McReaper/InternshipINFO4}{https://github.com/McReaper/InternshipINFO4}
\end{center}

\subsection{Documentation du projet}

<expliqué ce qui a été fait sans rentrer dans trop de détails>

\subsection{Restructuration du code source du compilateur} 
\label{warnings}

\subsection{Implémentation des politiques sous forme de modules Linux}

\subsection{Tests d'analyse de performances}
