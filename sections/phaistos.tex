
\section{PhaistOS, le générateur d'ordonnanceurs d'E/S}
\label{phaistos}

``PhaistOS'' est le nom d'un Langage Dédié, ou Domain Specific Language (DSL) 
en anglais. Utilisé dans le domaine de l'ordonnancement des entrées/sorties (E/
S) d'un système d'exploitation (SE), il a pour but de faciliter la conception 
et la validation de politiques d'E/S écrite par un utilisateur. Il est le 
fruit issu du projet VeriAMOS ayant pour but de s'attaquer au problème de vérification d'une classe de services de systèmes d'exploitation.

\subsection{Pourquoi PhaistOS ?}
\label{why_phaistos}

Il faut tout d'abord expliquer pourquoi VeriAMOS à lieu d'être. En effet ce 
projet à pour but d'améliorer la vérification de systèmes d'exploitation, qui 
reste aujourd'hui compliqué pour deux raisons principales. Premièrement, les 
propriétés mathématiques des algorithmes sous-jacents sont généralement 
difficiles à exprimer, deuxièmement, la structure du code qui implémente ces 
algorithmes dans le SE est en général très complexe, comme par exemple avec 
Linux, écrit en C. 

Afin d'éviter de se retrouver dans cette situation d'un niveau de difficulté 
trop élevé pour effectuer une vérification efficace, VeriAMOS propose 
d'utiliser une technique d'abstraction basée sur un DSL (comme le fait 
Bossa~\cite{Barreto-Muller:asf2002}), ainsi qu'un analyseur statique, qui, à 
froid, pourra vérifier un ensemble de services fournis par le SE, comme par 
exemple celui de l'ordonnanceur des requêtes d'E/S.

Dans cette approche de développement système, le programme écrit avec le 
langage du DSL pourra être compilé pour générer un code source dans le langage 
du système et qui implémente le service visé (plus de détails Partie~\ref
{context}). En limitant ainsi la capacité d'expression du langage, le DSL 
contraint le programmeur et permet d'éviter une mauvaise utilisation des outils 
du langage cible, permettant ainsi d'améliorer la robustesse des services 
concernés du SE. Le développeur pourra donc se focaliser sur l'écriture de sa 
politique système de haut niveau et déléguera la génération de code au 
compilateur du DSL.

VeriAMOS offre une opportunité de formaliser et d'implémenter une nouvelle 
approche à la vérification de systèmes d'exploitation, sur différents services. 
Ici nous nous focaliserons sur celui de l'ordonnanceur d'E/S du système, mais 
nous pensons que cette approche reste générale et pourrait s'appliquer sur tout 
types d'ordonnanceurs, de systèmes de fichier, etc.

\subsection{Mon rôle dans ce projet}
\label{mission}

Pour résumer, PhaistOS se décompose donc en deux parties : le compilateur et 
l'analyseur, ma mission portant uniquement sur le compilateur, ou autrement 
dit, le générateur de code C. 

En effet, le but de ma mission était de faire évoluer le compilateur actuel 
pour qu'il puisse produire un code qui se rapproche de celui d'un module Linux.
Un module Linux étant une extension du noyau du système qui peut être branchée
et débranchée à volonté, à chaud, c'est-à-dire pendant que le système est 
chargé en mémoire et s'éxecute. À cet objectif se rajoute celui de tests 
d'analyses comparatives (qu'on appellera ``benchmark'' par la suite) ayant pour 
but de montrer que l'utilisation d'un DSL n'a que très peu d'impact sur les 
performances finales du code généré. Bien évidemment, d'autres tâches me seront 
assignées, comme celle de la documentation, ou de la participation à la 
rédaction du papier scientifique portant à décrire le fonctionnement de 
PhaistOS (plus de détails sur ma contribution au projet dans la Partie~\ref
{contrib}).

\subsection{Pré-requis}

Avant de commencer ma mission, il a fallut que je me familiarise avec mon 
environement de travail, surtout avec les outils que j'allais utiliser. Pour 
comprendre et bien aborder ce qui allait suivre, j'ai du lire des extraits de 
documentation, des pages de wikis, et même des fois des forums.

Tout d'abord, comprendre comment les ordonnanceurs d'E/S fonctionnaient sous 
Linux était la priorité. À savoir, que chaque ordonnanceur est considéré comme 
un module (intégré de base dans le SE ou non), et que l'ordonnanceur actif est 
contenu dans un fichier particulier \cite{IOSchedulers} du système (changer le 
contenu du fichier change l'ordonnanceur actif).

Il a fallut ensuite que je me familiarise avec l'environement de travail 
virtuel qui m'était proposé, c'est-à-dire ce qui existait déjà sur le projet, 
comprenant la documentation, les exemples, le moyen employé pour implémenter 
PhaistOS dans le système et pour finir le code source de PhaistOS des 
différents dépôts de système de versions Git hébergés sur la plateforme \href
{https://gitlab.inria.fr/}{Gitlab}.

Plus tard, j'ai du utiliser un réseau de serveurs français dédié à la recherche 
appelé Grid'5000 pour effectuer des tests sur des machines plus performantes 
que ma machine personnelle. Pour cela j'ai du respecter une charte 
d'utilisation et apprendre les commandes propres à l'utilisation de la 
plateforme. À ce niveau là du stage, une connaissance fondamentale des systèmes 
d'exploitation était nécessaire pour chaque installation sur les machines 
distantes.

Et pour finir, je devais maîtriser et mettre en place des outils de benchmark 
que nous avions désignés avec mon tuteur. 