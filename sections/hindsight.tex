\section{Rétrospection}
\label{hindsight}

\subsection{Travail inachevé}

Si j'avais pu continuer à travailler sur le compilateur PhaistOS, je pense qu'un grand nombre d'améliorations auraient pu lui être apportées. Cependant le temps et la priorité des tâches ont fait que certains objectifs fixés pendant le stage non pas pu être réalisés. Voici une liste non-exhaustive de ces idées d'améliorations qu'on aurait bien aimé implémenter avec Nicolas :

\begin{enumerate}
    \item Créer un dossier de test avec plusieurs politiques pour pousser les 
    analyseurs dans leur retranchement, testant ainsi leur robustesse.
    \item Créer une preuve de concept visant à montrer que le visiteur de 
    modèle et la création d'un visiteur pour l'AST n'étaient pas nécessaire; et 
    que l'on pouvait remplacer ces visiteurs à l'aide d'une bibliothèque~\cite
    {visitors2021manual}, 
    déléguant ainsi une grande charge de travail.
    \item Régler tout les soucis de compilation ocaml présent dans les 
    différents dossier, comme cela a été fait avec \texttt{Static-Analysis}.
    \item Créer de nouvelles politiques avec PhaistOS, basées sur celles qui 
    existent déjà, pour pouvoir tester les limites du DSL plus en profondeur, 
    et explorer les nouvelles pistes d'améliorations.
    \item Changer le code du point d'entrée des programmes ocaml pour offrir 
    une interface en ligne de commande complète (\texttt{--help}, \texttt{-o}, 
    etc.).
    \item Étendre la batterie de tests de performances avec d'autres outils, 
    comme NAS, Phoronix, Parsec, Sysbench, etc.
\end{enumerate}

Cependant, ce n'est pas la priorité du chercheur de rendre le code plus 
qualitif ou robuste. Par contre, toutes ces idées pourraient faire l'objet d'un 
futur stage, facilitant ainsi le travail du chercheur sur la partie importante 
du projet de recherche. 

\subsection{Gestion de projet}

Au début du stage je n'avais pas accès aux locaux du bâtiment d'accueil, je 
travaillais donc depuis chez moi. Je passais des appels avec mon tuteur sur la 
plateforme bbb de l'UGA et on discutait de l'environement de travail et des 
premières tâches à réalisées. L'outil Slack nous aura aussi bien aidé pour 
apporter une aide à l'écrit, des fois nécessaire quand on veut partager du 
texte.

Les tâches principales étaient déjà définies mais restaient larges (Etude de 
l'existant, améliorations du compilateur, tests de performances), je les ai 
donc sous-divisées. J'ai adapté mes travaux au fur et à mesure, et pour 
illustrer ça, on peut comparer le planning de travail estimé au réel à travers 
les Figure~\ref{fig:gantt_1}~et~\ref{fig:gantt_2}, disponibles en annexe.

<RETOUR SUR LES GANTT>

Il est à noter qu'à partir du 1er juillet, je pouvais me rendre trois fois par 
semaine au bureau, ce qui a grandement accéléré les choses. Changer 
d'environement de travail m'a bien aidé car rester chez soi devant l'ordinateur 
n'était pas motivant, et voir les collègues de l'école au bureau rendait le 
travail plus amusant.

\subsection{Bilan personnel}

Je pense que durant mon stage, j'ai été capable de voir les choses d'un point 
de vue plus global, plus réfléchis, contrairement à mon ancien stage de fin de 
DUT où j'avais le nez dans le code sans trop me préoccuper à comprendre 
l'environement et le contexte. 

\underline{\texttt{EE6} :} Je pense que j'ai été capable de trouver 
l'information qu'il me fallait quand je bloquais, que pour trouver cette 
information je posais les bonnes questions (que ce soit sur Google où à 
Nicolas) et que j'étais capable de les exploiter sans avoir besoin d'aide 
extérieure. On peut penser aux modules récursifs dans la Partie~\ref{solution} 
où au script de mise en place d'une machine virtuelle Partie~\ref{qemu} où 
Nicolas m'a bien aidé concernant les outils à utiliser.

\underline{\texttt{EE4.4} :} J'ai aussi eu l'occasion de documenter et 
d'expliquer le compilateur à Nicolas qui n'avait pas eu le temps de gagné en 
expertise sur le projet auparavant. Cette capacité à présenter, documenter et 
même rédiger sur le compilateur a été une partie importante du stage car j'ai 
pu avoir l'opportunité de participer à l'écriture d'un papier scientifique 
rédigé en anglais concernant l'outillage du compilateur PhaistOS.

Grâce à cette expérience qui m'a été offerte par M. Palix, j'ai pu découvrir ce 
qu'était le travail dans le domaine publique. Cette opportunité m'aura beaucoup 
apportée, en terme de connaissance comme d'expérience, et j'en suis très 
reconnaissant. Il y a 2 ans déjà, j'avais réalisé un stage dans le domaine 
privé, dans l'entreprise Atos à Grenoble. Ces deux expériences m'ont permis de 
réaliser les différences significatives qu'il existait entre les deux domaines. 
Malheureusement, l'année prochaine je n'aurai pas l'opportunité de pouvoir 
choisir de retourner dans le domaine publique, car d'après la charte de l'école 
concernant les stages, une seule expérience en laboratoire est autorisée. Cela 
dit, j'espère avoir un choix large concernant mon prochain stage, et de pouvoir 
retravailler dans une atmosphère saine l'an prochain.

\subsection{Le futur de PhaistOS}

Concernant PhaistOS, il est évident que ce projet est original et a un 
potentiel d'exploitation dans le futur. Assez maléable et relativement simple à 
maintenir, il permettra d'écrire ses propres politiques d'E/S pour son système 
et offre des possibilités d'optimisation intéressante côté applicatif. Le 
programme en cours d'utilisation pourra potentiellement changer d'algorithme 
d'ordonnancement à chaud, pour optimiser son débit et sa latence d'E/S sur le
(s) disque(s). Cette possibilité reste donc séduisante même si le DSL reste 
restrictif concernant son API et ses possibilités de personnalisation, c'est le 
parti pris pour la simplification offerte par le langage.
